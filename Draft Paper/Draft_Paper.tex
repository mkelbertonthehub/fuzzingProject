% TEMPLATE for Usenix papers, specifically to meet requirements of
%  USENIX '05
% originally a template for producing IEEE-format articles using LaTeX.
%   written by Matthew Ward, CS Department, Worcester Polytechnic Institute.
% adapted by David Beazley for his excellent SWIG paper in Proceedings,
%   Tcl 96
% turned into a smartass generic template by De Clarke, with thanks to
%   both the above pioneers
% use at your own risk.  Complaints to /dev/null.
% make it two column with no page numbering, default is 10 point

% Munged by Fred Douglis <douglis@research.att.com> 10/97 to separate
% the .sty file from the LaTeX source template, so that people can
% more easily include the .sty file into an existing document.  Also
% changed to more closely follow the style guidelines as represented
% by the Word sample file. 

% Note that since 2010, USENIX does not require endnotes. If you want
% foot of page notes, don't include the endnotes package in the 
% usepackage command, below.

% This version uses the latex2e styles, not the very ancient 2.09 stuff.
\documentclass[letterpaper,twocolumn,10pt]{article}
\usepackage{usenix,epsfig}
\begin{document}

%don't want date printed
\date{}

%make title bold and 14 pt font (Latex default is non-bold, 16 pt)
\title{\Large \bf Webcam Fuzz Testing: Testing IoT Deployments}

%for single author (just remove % characters)
\author{
{\rm Matthew Elbert}\\
University of Utah
\and
{\rm Jeffrey Kitchen}\\
University of Utah
% copy the following lines to add more authors
% \and
% {\rm Name}\\
%Name Institution
} % end author

\maketitle

% Use the following at camera-ready time to suppress page numbers.
% Comment it out when you first submit the paper for review.
\thispagestyle{empty}


\subsection*{Abstract}
Your Abstract Text Goes Here.  Just a few facts.
Whet our appetites.

\section{Introduction}
\subsection{Motivation}

Fuzz testing is a method of testing in which random, potentially invalid inputs are sent to a program or service in order to test for vulnerabilities. It is often used as either a gray or black-box testing platform for security testing. Fuzz testing can be a very cheap and effective implementation of testing as many inputs can be generated, sent, and analyzed automatically, without the need for a human to monitor. However, a truely random test input could easily not provide anything useful in a reasonable amount of time. Therefore, many solutions (citations needed) propose a different method where random mutation of good inputs is used so that requests that are close to being correct are used. 
Because the world has become more internet-connected than ever, fuzz testing can be an important tool for developers as network-facing interfaces can experience any input, and need to be tested for this randomness. However, many functions are very state-driven, and a true, random test may only test a single interface, and not the whole system. Because of this, many of these fuzz testers must be stateful in order to do a complete test. Because of many security issues with the current Internet architecture, there are obviosuly some security measures put in place to almost anything that is deployed on a network, even if it is meant to only be local. Therefore, something as simples as loggin into an interface needs to be taken into account in order to et more meaningful tests.

\subsection{Problem}

In this project, we designed, implemented and evaluated an automated tester for a networking service. This is meant to evaluate the value of fuzz testing on a network, with the goal of hopefully finding a previously undiscovered bug. A very basic and highly proliferated technology deployed on the internet is an HTTP server. But because of this prevalence, popular servers such as Apache have been thoroughly tested and documented. However, more and more devices are being connected, with an uptake in the Internet of Things (IoT) mentality. Therefore, we chose wireless IP cameras as the platform to fuzz test. These are some of the most widely available IoT devices for consumers. We found that the two devices we purchased ran basic HTTP servers, and we could use tools made for these applications. 


\section{Related Work}

We looked at papers like SecFuzz~\cite{secfuzz} and SNOOZE~\cite{snooze} as well as tools like AutoFuzz~\cite{autofuzz}

\section{Proposed Solution}

Use solutions for HTTP fuzzing, e.g. Pathoc, to fuzz the HTTP servers running on webcams.

\section{Implementation}

We used Pathoc for crafted malice towards the mini\_httpd and DLink servers. Pathoc has a built in function to do many requests at one time, but we found that this method had all of the GET requests on a single TCP connection, and therefore made it more difficult to diagnose in programs such as Wireshark. Therefore, since Pathoc can easily be implemented using the command line interface, a shell script was used as a wrapper that would invoke Pathoc multiple times, giving a new HTTP request everytime. In this way, we could also control fairly easily the commands for Pathoc to run, instead of having a single command run multiple times.

\section{Results}

{\footnotesize \bibliographystyle{acm}
\bibliography{biblio}}




\end{document}






